\documentclass[fleqn]{article}

\usepackage{polski}
\usepackage[utf8]{inputenc}
\usepackage[polish]{babel}
\usepackage{parskip}
\usepackage{icomma}
\usepackage[a4paper,includeheadfoot,margin=1.27cm]{geometry}
\usepackage{float}
\usepackage{graphicx}
\usepackage{amsmath}
\usepackage[hypcap=true]{subcaption}
\usepackage{xcolor}
\usepackage{transparent}
\usepackage{listings}
\usepackage[colorlinks=true, linkcolor=blue, pdfborder={0 0 0}]{hyperref}

\renewcommand\thesection{\arabic{section}.}
\renewcommand\thesubsection{\alph{subsection})}
\renewcommand\thesubsubsection{}
\newcommand\square[1]{
	\fcolorbox{black}{#1}{\rule{0pt}{6pt}\rule{6pt}{0pt}}
}

\brokenpenalty=1000
\clubpenalty=1000
\widowpenalty=1000

\title{\textbf{STP} \\ \large Projekt II - Zadanie 2.39}
\author{Marcin Skrzypkowski \\ nr albumu 283419}

\begin{document}

\maketitle

\setcounter{page}{0}
\thispagestyle{empty}

\pagebreak
\setcounter{page}{1}


\tableofcontents
\pagebreak


\begin{center}

	Obiekt regulacji jest opisany poniższą transmitancją

	\Large{\textbf{G($s$)=}$\frac{K_oe^{-T_os}}{(T_1s+1)(T_2s+1)}$}
\end{center}

\begin{table}[H]
	\centering
	\label{my-label}
	\begin{tabular}{|l|l|l|l|l|}\hline
		% &K & T_1 & T_2 &   \alpha_1 & \alpha_2 & \alpha_3 & \alpha_4 \\ \hline
		% &$5.5$ & $7$ & $7$ &$-0.12$ & $0.4$ & $0.25$ & $0.2$\\ \hline
		$K_o$ & $T_o$ &$T_1$ &  $T_2$   \\ \hline
		$4.5$ & $5$ & $1.87$ &$5.31$\\ \hline
	\end{tabular}
	\caption{Wartości parametróþw obiektu regulacji}
\end{table}

\section{Wyznaczenie transmitancji dyskretnej}




\end{document}
